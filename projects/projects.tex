\documentclass{aa}
\input{preface}
\begin{document}


This is not yet a paper.  It is a collection of paper ideas and some 
preliminary analysis that will make it in to one or more papers.

Data summary: the 7m and 12m data have all been delivered, but the UV coverage
is non-optimal.  Importantly and unfortunately, no single-dish (total power)
data were ever acquired.  APEX data with SHFI have been acquired but not
reduced.  The original proposed project connecting cloud scales to core scales
will not be possible until the single dish data are completed.


Subprojects:
\begin{enumerate}
    \item Core identification and mass estimation and maybe even
        core mass function constraints.  Lead: Ginsburg.  This document
        is a quasi-draft of that work.
    \item ``A multi-phase outflow from a high-mass protostar"
        The Lacy jet is detected in both CO and RRLs.  This tells us something
        about its proximity to the HII region IRS2, but what else can we learn
        from it?  Lead: maybe Ginsburg.  Maybe this gets incorporated
        into other papers.
    \item Comparitive chemistry of the W51 cores.  Lead: Victor Rivilla \&
        Maite Beltr{\'a}n, possibly with participation from Alvaro
        Sanchez-Monge.
        This project will involve identifying all of the lines in the W51 cores
        and examining how the chemistry relates to physical parameters.
        We will use LTE modeling tools (XCLASS, MADCUBA) for many species to
        identify lines and determine abundances and temperatures
\todo{Discuss with Victor \& Alvaro: Can the moderate-mass cores be
chemically distinguished from the high-mass?  Do we know their masses well enough
now to do this?}
    \item \formaldehyde and CO turbulent cloud modeling / simulation comparison
        (Loughnane).  A difficulty here is that combination of the 7m and 12m
        data has not worked out very well yet, either for the lines or the
        continuum, and that may render cloud-scale analysis quite difficult.
    \item A study of the CO outflows.  Lead: Maybe Luke Maud?  Ciriaco Goddi
        will be involved.  The $^{12}$CO and SO outflows are spectacular and
        plentiful.  It is not obvious whether they can or should be
        incorporated into this paper; at least, I am not presently prepared to
        put in the effort to quantify the outflows properly.  Ciriaco is PI of
        a long-baseline program that has resolved the e2/e8 and North cores
        with $5\times$ better resolution than this program, and that data set
        may therefore be better suited to a core-outflow association work.
    \item Relative kinematics of ionized and molecular gas.  Lead: Galv{\'a}n-Madrid.
        Kinematics similar to \citet{Keto2008a} using RRLs+molecular lines to determine
        outflow vs infall
        
\end{enumerate}



\section{Overview}
\todo{This section is not part of any proposed paper text.}
The work below will be incorporated into one or more papers as it proves useful.
The figures \& text represent an initial exploration of the data.  There are some important notes
that are not included in the text:
\begin{itemize}
    \item The data reduction has some substantial problems because of strange UV coverage
        and stranger behavior of clean (for which I have opened many tickets).  It appears
        impossible to get the noise below $\sim1 $ mJy around the bright sources, and detections
        are unreliable below $\sim10$ mJy in these area (which I've tried to account for when doing
        source extraction)
    \item The 7m data does not combine well with the 12m data.  It seems to
        cause major large-angular-scale artifacts.  I think this is a weighting
        issue in which the 7m data have lower noise than the 12m data and are
        therefore too heavily weighted, causing large "halos" where the large
        angular scale dominates over the small, providing spuriously strong
        detections.  In principle, this can be solved by downweighting the 7m data,
        but it turns out CASA does not have that implemented (\texttt{innertaper}
        is the relevant keyword)

\end{itemize}


\end{document}
