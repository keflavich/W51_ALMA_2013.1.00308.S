\begin{table*}[htp]
\caption{Spectral Lines in SPW 1}
\begin{tabular}{ll}
\label{tab:linesspw1}
Line Name & Frequency \\
 & $\mathrm{GHz}$ \\
\hline
H$_2$CO $3_{2,1}-2_{2,0}$ & 218.76007 \\
HC$_3$N 24-23 & 218.32471 \\
HC$_3$Nv$_7$=1 24-23a & 219.17358 \\
HC$_3$Nv$_7$=1 24-23a & 218.86063 \\
HC$_3$Nv$_7$=2 24-23 & 219.67465 \\
OCS 18-17 & 218.90336 \\
SO $6_5-5_4$ & 219.94944 \\
HNCO $10_{1,10}-9_{1,9}$ & 218.98102 \\
HNCO $10_{2,8}-9_{2,7}$ & 219.73719 \\
HNCO $10_{0,10}-9_{0,9}$ & 219.79828 \\
HNCO $10_{5,5}-9_{5,4}$ & 219.39241 \\
HNCO $10_{4,6}-9_{4,5}$ & 219.54708 \\
HNCO $10_{3,8}-9_{3,7}$ & 219.65677 \\
CH$_3$OH $8_{0,8}-7_{1,6}$ & 220.07849 \\
CH$_3$OH $25_{3,22}-24_{4,20}$ & 219.98399 \\
CH$_3$OH $23_{5,19}-22_{6,17}$ & 219.99394 \\
C$^{18}$O 2-1 & 219.56036 \\
H$_2$CCO 11-10 & 220.17742 \\
HCOOH $4_{3,1}-5_{2,4}$ & 219.09858 \\
CH$_3$OCHO $17_{4,13}-16_{4,12}$A & 220.19027 \\
CH$_3$CH$_2$CN $24_{2,22}-23_{2,21}$ & 219.50559 \\
Acetone $21_{1,20}-20_{2,19}$AE & 219.21993 \\
Acetone $21_{1,20}-20_{1,19}$EE & 219.24214 \\
Acetone $12_{9,4}-11_{8,3}$EE & 218.63385 \\
H$_2$$^{13}$CO $3_{1,2}-2_{1,1}$ & 219.90849 \\
SO$_2$ $22_{7,15}-23_{6,18}$ & 219.27594 \\
SO$_2$ $v_2=1$ $20_{2,18}-19_{3,17}$ & 218.99583 \\
SO$_2$ $v_2=1$ $22_{2,20}-22_{1,21}$ & 219.46555 \\
SO$_2$ $v_2=1$ $16_{3,13}-16_{2,14}$ & 220.16524 \\
\hline
\end{tabular}
\par
The Categories column consists of three letter codes as described in Section \ref{sec:contsourcenature}.In column 1, \texttt{F} indicates a free-free dominated source,\texttt{f} indicates significant free-free contribution,and \texttt{-} means there is no detected cm continuum.In column 2, the peak brightness temperature is used toclassify the temperature category.\texttt{H} is `hot' ($T>50$ K), \texttt{C} is `cold' ($T<20$ K), and \texttt{-} is indeterminate (either $20<T<50$K or no measurement)In column 3, \texttt{c} indicates compact sources, and \texttt{-} indicates a diffuse source.
\end{table*}
